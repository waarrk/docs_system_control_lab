\documentclass[a4paper]{jarticle}
\usepackage{proposal_ec}
\begin{document}
\氏名{鷲尾 優作}			%% 自分の氏名
\学籍番号{331124}			%% 学籍番号
\指導教員{佐藤 拓史}			%% 指導教官名

%% 以下の各項目の文中に改行を入れたい場合は,'\\' を挿入する。
%% 単に空行を入れるとエラーになります。
%
% ----- 研究テーマ -------------------------------------------%
%
\研究テーマ{
  積雪寒冷地域における自動運転を想定した車轍検出システムの構築
}
%
% ----- 研究の背景・特色な点 -------------------------------%
%
\研究の背景・特色{
  現在,各国で自動運転の実用化に向けた開発競争が激化している.
	しかしながら,現行のシステムは利用者の多い都市部を中心に実験・データ収集を実施している為,
	積雪寒冷地域において自動走行の精度が低下する問題が指摘されている(積雪寒冷地未対応課題)\cite{tesla}\cite{jigyo}.

	加えて,自動運転の有無にかかわらず冬季は凍結,積雪に起因する「冬型事故」が多発する\cite{hokkaido}ため,この対策が不可欠である.

	そこで本研究では,冬型事故のひとつである「わだち事故」を想定し,
	積雪寒冷地域における事故等軽減を目的とした車両走行安全性向上システムの構築を目指す.
	
	この「車轍検出システム」は「走行可不可判定」および「車両滑り危険性判定」の2点から成る.
	車轍は,車輪の軌跡によって形成される走行した跡を指す.
	走行方向に対して車轍が斜めになると,ハンドルを取られスリップの原因になるほか,積雪深によっては直接の運転を困難とし,交通渋滞を招く.
	今回,これを検出し自動ないしは半自動にて,車両の侵入を制限したり,車両の減速を促すことは,「わだち事故」低減に効果的であると推定される.

	積雪寒冷地域においては,積雪状況の変化によって周辺状況が変化するため,事前の学習データに依存しない設計が求められる.
	現在,カメラを用いたNeRF(Neural Radiance Fields)やLiDARを用いた点群の3次元復元技術が向上しているため,これらを活用した路面情報のリアルタイム取得を前提とした設計を行う.

	システムの開発にあたっては,研究期間中に積雪がある可能性が低いため,実際の車両に搭載することは困難である.
	そこで,本研究では,小型の車両モデルを用いて,車両の走行軌跡を検出することを目指す.
	走行する車轍は,実際の形状を模した簡易的なフィールドを作成するものとする.

	検出プログラムはROS2 Humbleのノードとして動作するものとし,Jetson Nano上でDockerコンテナとして運用する.検出器にはRealSenseD435を想定している.
}
%
% ----- 研究目的 ---------------------------------------------%
%
\研究の目的{
	積雪寒冷地域における事故等軽減
}
%
% ----- 研究計画・方法 ---------------------------------------------%
%
\研究計画・方法{
\textbf{A.各種動作確認\\}
Jetson Nano上に構成したROS2環境で,RealSenseD435を含む各機能の動作確認を行う.

\begin{itemize}
	\item Rviz2等可視化ツールの準備 DDS通信の確認
	\item その他使い方の確認
\end{itemize}

\textbf{B.必要資材の選定\\}
研究に入るにあたって必要な資材をリストアップする.

\begin{itemize}
	\item 車両モデル
	\item フィールド
\end{itemize}

\textbf{C.車両モデル 実験フィールドの用意\\}
実験に使用する車両モデルは,小型の4輪車両モデルを想定する.
サイズについては,実験室内での走行を想定しているため,実験室内の障害物を回避できる程度の大きさとする.(検討中)

\begin{itemize}
	\item 車両の仕様決定
	\item 車両の設計
	\item フィールドの作成
\end{itemize}

\textbf{D.制御実験\\}
以下のような項目を試す.
轍検出は,先に比較的簡単と思われる.セグメンテーションによる路面のラベリングと,ハフ変換による軌跡の抽出で実装する.
終了後,点群を用いた軌跡検出にトライする.

\begin{itemize}
	\item 画像のみを使った走行可能場所判定
	\item RealSenseの深度データを使ってみる
	\item 上記データにハフ変換などの処理を試す
\end{itemize}

\textbf{E.資料作成\\}
追実験可能なように,資料を作成する.

\begin{itemize}
	\item ROSやDockerについての資料
	\item ハードウェアの資料
	\item 制御についての資料
\end{itemize}

\textbf{F.卒業論文作成\\}
以上のことをまとめ,卒業可能にする.

\begin{itemize}
	\item 卒業論文作成
	\item 発表資料作成
\end{itemize}
}
%
% ----- 参考文献 ---------------------------------------------%
%
\参考文献{
%---------------------------------------------------------------%
\renewcommand{\refname}{}
\vspace{-18mm}
\begin{thebibliography}{99}
%---------------------------------------------------------------%
\bibitem{tesla} Detroit Tesla (YouTube): 
	``SUPER SCARY FSD BETA 10.69.25 Detroit MI'', \\
	https://youtu.be/\_swe1pXXgNo (2022/12/24).
\bibitem{jigyo} 北海道経済産業局: 
	``積雪寒冷地域の交通弱者移動支援のための雪道走行を可能とする
	自動運転技術の開発」'', 
	https://www.chusho.meti.go.jp/keiei/sapoin/portal/seika/2017/2910102005h.pdf (令和2年5月).
\bibitem{hokkaido} 北海道警察本部交通部交通企画課: 
	``吹雪など視界不良時における交通事故の実態'', 
	https://www.police.pref.hokkaido.lg.jp/info/koutuu/fuyumichi/blizzard.pdf 
	(令和3年).
%---------------------------------------------------------------%
\end{thebibliography}
}

\フォーム作成

\end{document}