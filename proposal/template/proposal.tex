\documentclass[a4paper]{jarticle}
\usepackage{proposal_ec}
\begin{document}
\氏名{システム制御 研太郎}			%% 自分の氏名
\学籍番号{293XXXX}			%% 学籍番号
\指導教員{佐藤 拓史}			%% 指導教官名

%% 以下の各項目の文中に改行を入れたい場合は、'\\' を挿入する。
%% 単に空行を入れるとエラーになります。
%
% ----- 研究テーマ -------------------------------------------%
%
\研究テーマ{
  現時点でふさわしいと思う研究のテーマを記す
}
%
% ----- 研究の背景・特色な点 -------------------------------%
%
\研究の背景・特色{
参考文献を引用して,研究の背景を述べる.
ただ調べたことを並べるだけではなく,最終的に研究の特色や目的につなげるものを選択する必要がある.

研究の背景から自分の研究をどのように進めるのか,
何が特色になるのかを述べる(この部分が重要).
他の研究機関で行っているような後追いはあまり意味がないので,違いを見出すことが求められる.
}
%
% ----- 研究目的 ---------------------------------------------%
%
\研究の目的{
研究の背景・特色から自分の研究の目的を記す
}
%
% ----- 研究計画・方法 ---------------------------------------------%
%
\研究計画・方法{
研究の目的を達成させるためには何を解決しなければならないかを考え,項目を立てる.

その項目について,どんな方法で進めていくのか,何を用いるのかなどの詳細を考える.
}
%
% ----- 参考文献 ---------------------------------------------%
%
\参考文献{
%---------------------------------------------------------------%
% ----- 作成時には次の一文は削除するかコメントアウト ----------%
文献調査した文献を挙げることになるが,研究の背景・特色等で引用するものを挙げる.
%---------------------------------------------------------------%
\renewcommand{\refname}{}
\vspace{-18mm}
\begin{thebibliography}{99}
%---------------------------------------------------------------%
\bibitem{ref1} 著者: 
	``タイトル'', 
	出展元, Vol.XX, No.X, 
	pp.XXX-XXX (発行年).
%---------------------------------------------------------------%
\end{thebibliography}
}

\フォーム作成

\end{document}