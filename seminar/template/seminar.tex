%---------------------------------------------------------------%
\documentclass[a4paper]{jarticle}
%---------------------------------------------------------------%
% プリアンブル
%---------------------------------------------------------------%
\usepackage{rpstyle}
\usepackage{amsmath,amsthm,amssymb}
\usepackage{wrapfig}
\usepackage{epic, eepic}
\usepackage{ascmac}
\usepackage{verbatim}
\usepackage{multicol}
\usepackage{subfigure}
%\usepackage{arydshln}
% dvioutで確認する場合は以下を有効にする
\usepackage[dvips]{graphicx,color}
% pdf化する場合は以下を有効にする
%\usepackage[dvipdfmx]{graphicx,color}

\def\degC{\kern-.2em\r{}\kern-.3em C}
\def\degree{\kern-.2em\r{}\kern-.3em}
%---------------------------------------------------------------%%
% タイトル&著者
\title{ゼミ資料 テンプレート}
\author{システム制御 研太郎}
%
% 報告日
\date{June 24, 2020}
%
% 概要
\abstract{
ゼミ資料の要約を記入する.この資料が何をまとめたものかが読んで分かるようにまとめる.
}
%
% キーワード
\keyword{キーワード1, キーワード2, キーワード3, $\cdots$ \\ (ゼミ資料で重要となるキーワードを挙げておく)}
%
% 書式の再設定
\newcommand{\sline}{\rule{17.5cm}{0.5mm}}
%
\begin{document}
\maketitle
%
%---------------------------------------------------------------%
\section{はじめに}
%
ここでは,このゼミ資料をまとめたいきさつや何について報告しているのかを述べる.
アブストと重複することもあるが,手を抜かずに書くこと.
%
%---------------------------------------------------------------%
\section{報告内容}
%
ここでは,ゼミ資料としてまとめる内容を書く.

章題などはよく考えて付けること.場合によっては節や小節などを適宜用い,何を伝えたいのかが分かる構成とする.
まとめる際には,最終的な卒業論文を意識して,細かく丁寧に書く.
知らない人に分かってもらう説明を意識して,くどいくらいでちょうどよいものになる.

式導出では途中を省略せず,しっかりと行間を埋めるようにする.図や表などの作成は時間をかけることを意識する.
設定や環境構築などをまとめる場合は,画面キャプチャなどをうまく用いて,後で同じことが容易に実現できるようにすることを意識する.
%
%
%---------------------------------------------------------------%
\section{おわりに}
%
最後にこのゼミ資料をまとめる結言を書く.参考文献がある場合は,最後に列記しておくこと.
%
%---------------------------------------------------------------%
\begin{thebibliography}{99}
  \bibitem{ref1}
    著者:論文題目,掲載論文誌名,号巻,掲載ページ,学会名,発行年
  \bibitem{ref2}
    著者:本のタイトル,参照ページ,出版社,発行年
\end{thebibliography}
%
\end{document}
